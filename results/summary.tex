\section{Discussion}

TALK ABOUT SNOW ALSO

Weather stations and satellites continuously monitor weather across the globe and the resulting data is made freely available. Out system employs the same data as input, which makes it very easy to model any type of environment. Finding and specifying the input data to model these two environments, for example, took less that ten minutes. The terrains used in this chapter are not taken from real world data of the location on which climate data is being modelled, however, but from a pool of pre-generated terrains. Finding and specifying the necessary input data increases when factoring in the time it takes to find and generate this necessary height-map data.\\
The least intuitive input is the soil infiltration rate as this necessitates that the user knows the correlation between soil type and soil infiltration rate. This could be improved, however, by providing a list of soils to the user and automating the correlation.\\

The water networks that the system generates accurately reflect the variation in precipitation quantity and intensity. This is apparent as the networks vary in size and depth on a monthly basis as the amount of calculated standing water varies. In addition, the water networks are generated very quickly for all twelve months and the water flow simulation is rendered in real-time, promoting user interaction. \\

Similarly, the clustering algorithm also runs in real-time, permitting users to quickly visualise the impact of varying the number of clusters. However, as the number of clusters increases, identifying the different clusters on the terrain becomes difficult and the associated cluster data displayed to the user, overwhelming. A solution to this would be to automatically determine a suitable number of clusters based on terrain resource variability and a more intuitive "level of realism" configuration parameter. By doing so, the clustering element can be bypassed entirely from a user perspective and processed in the background.\\

In terms of vegetation, the system accurately determines the species suitability to each cluster in terms of slope, temperature and soil moisture and effectively communicates this to the user during the species selection phase (see section \ref{sec:plant_suitability_filtering}). Note that sun exposure is not considered at this stage as it varies throughout a simulation as plants grow and project shade on the undergrowth. This suitability data is subsequently used to prevent ill-suited species from being inserted into a given ecosystem simulator run. This accelerates the time it takes to generate vegetation distributions as simulations in which all species are ill-suited are identified beforehand and skipped entirely. \\
By analysing the average size and instance count of individual species produced by the ecosystem simulator for each cluster, it is apparent that their vigour reflects their suitability to each cluster environment. Species present, as well as available resources, have a big impact on resulting distributions. This is made apparent in the tropical rainforest tests, for example, where shade-loving plants only grow in clusters suited to canopy plants, on which they depend for vital shade coverage. It is also illustrated when comparing the results of the two rainforest tests (sections \ref{sec:results_trop_rainforest} and \ref{sec:results_trop_rainforest_big}). More species are present in the second test and, as a consequence, inter plant competition is more intense. This has a big impact on resulting plant distributions. In cluster four, for example, the added competition of other large canopy plants causes the number of Brazil Nut plants to decrease by approximately eighty percent. Their average size increases by over forty percent, however, as the fierce competition makes it difficult for smaller, less vigorous plants to survive. The system reproduces the concept of survival of the fittest.\\

Finding the necessary configuration data for plant species can be difficult. Although optimal water, sun exposure and temperature requirements can be found relatively easily, determining the lower and upper limits is more challenging. A solution to this would be to model plant functional types \cite{Moncrieff2015} rather than individual plant species. Plant functional types encompass a number of species and corresponding resource requirement data can be preconfigured with the help of biologists. Modelling individual plant species could then be done by the user using the parent functional type as a template for parameter specification.\\
