\chapter{Results and Discussion}

This work focuses on river networks (section \ref{sec:rivers_and_streams}, water reserves (section \ref{sec:water_bodies}) and vegetation (section \ref{chap:vegetation}) when generating realistic virtual rural worlds. An essential intermediary in doing so, being clustering (section \ref{chap:clustering}). Because the results and efficiency of these individual constituents are analysed and discussed in detail in the main body of this document, this chapter focuses more on the aggregate results of all systems parts working together. \\

To do so, three terrains taken from the U.S. Geological Survey \protect\footnotemark \footnotetext{\url{http://www.usgs.gov}} are loaded and resources specified to model two distinct environments: tropical rainforest and alpine. When modelling these environments, temperature and rainfall data are specified using using freely available weather data at locations with such climates \protect\footnotemark \footnotetext{\url{http://weather-and-climate.com}}. Given this, a set of five species that are suited to the given environment are configured using freely available online species data  \protect\footnotemark \footnotetext{\url{http://www.usgs.gov}} and a valid distribution created using the ecosystem simulator alongside the clustering algorithm configured with a fixed cluster count of ten. The resulting water networks and cluster plant distributions are subsequently analysed. Specifying resources to generate the virtual worlds in this chapter took less than 10 minutes for all scenarios. It is important to note that these tests do not attempt to reproduce real-world plant distributions given identical resources but rather produce plausible distributions with minimal effort from the user. \\

To end this chapter, the strength and weaknesses of the system in terms of the generated results are discussed.\\
