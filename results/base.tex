\chapter{Results and Discussion} \label{chap:results}

This work focuses on river networks (section \ref{sec:rivers_and_streams}), water reserves (section \ref{sec:water_bodies}) and vegetation (section \ref{chap:vegetation}) in order to generate realistic virtual rural worlds. Clustering is an achieving this (section \ref{chap:clustering}). Because the results and efficiency of these individual constituents are analysed and discussed in detail in the main body of this document, this chapter focuses more on the aggregate results of all system parts working together. To do so, three terrains taken from the U.S. Geological Survey \protect\footnotemark \footnotetext{\url{http://www.usgs.gov}} are loaded and resources specified to model two distinct environments: tropical rainforest and alpine. When modelling these environments, temperature and rainfall data are specified using freely available weather data at locations with such climates \protect\footnotemark \footnotetext{\url{http://weather-and-climate.com}}. Given this, a set of five species that are suited to the given environment are configured using freely available online species data  \protect\footnotemark \footnotetext{\url{http://davesgarden.com/guides/pf/}} and a valid distribution created using the ecosystem simulator alongside the clustering algorithm configured with a fixed cluster count of ten. The resulting water networks and cluster plant distributions are subsequently analysed. Specifying resources to generate the virtual worlds in this chapter took less than ten minutes for all scenarios.  \\

It is important to note that these tests do not attempt to reproduce real-world plant distributions given identical resources but rather produce plausible distributions given the properties of the generated clusters and configured species. In order to do so, suitability analysis is performed on each resulting specie and cluster pair. The cluster count is fixed and the number of species, small, in order to keep this analysis data manageable and efficiently communicable. \\

In order to show the scalability of the system, a third test is performed similar to the tropical test but with a larger number of species (fifteen). Because of the large amount of data which results from this test, however, detailed analysis of each specie and cluster pair is not performed. Instead, we briefly discuss notable characteristics of each cluster.\\

At the end of this chapter, the strength and weaknesses of the system in terms of the generated results are discussed.\\

