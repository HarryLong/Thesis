\section{Tropical rainforest with fifteen plant species}

Sections \ref{sec:results_trop_rainforest} and \ref{sec:results_alpine} focus on analysing, in detail, the species content of each individual cluster. Because of the level of detail necessary for this analysis and to keep information communicable, the number of species used is limited to five. The system should be scalable, however, and able to cater for a larger number of species. This is the focus of this chapter. To do so, climate data identical to the tropical test run is configured (see table \ref{tab:results_tropical_input_resources}), the same number of clusters produced (ten), but a total of fifteen tropical plant species are configured and used when generating the vegetation distributions (refer to appendix \ref{AppendixE} and \ref{AppendixG} for species details). Because the configuration data and cluster count are identical to that of the tropical test, the clusters remain unchanged and are summarized in figure \ref{fig:results_tropical_cluster_hum_temp_illum} and table \ref{tab:results_tropical_cluster_slope_covarea}.\\

The instance count along with the minimum, maximum and average height of each specie and cluster pair is summarized in appendix ... . Similarly to the previous tropical test, these results show that: no plants are able to survive in clusters 3, 5 and 7 primarily because there is too little soil moisture; cluster 8 is ill-suited to all species because of too much soil moisture; cluster 10 is unsuited because the sun exposure is too little to enable canopy plants to grow but too high to permit shade-loving species to grow without vital shade cover of these canopy plants. Unlike the previous tropical test, however, the resources of cluster 9 are within the survival range of a plant species.\\

When discussing the species content of each cluster below, those present in the tropical test discussed previously (section \ref{sec:results_trop_rainforest}) are ignored as their cluster suitability remains unchanged. An important property which is noteworthy with these results, however, is that although the suitability of the pre-existing species to each cluster does not change, their instance count and size properties often do. This is a direct consequence of adding more species to the cluster window and, therefore, having to redistribute resources. This can work in the species favour, as is the case for Heliconia in cluster one for example. It can also be a disadvantage, as is the case for King of Bromeliads in cluster four, for example. It is difficult to pinpoint why adding a given species works as an advantage or disadvantage to another species, however, as the cause can be indirect. In clusters five and six, for example, shade-loving species King of Bromeliads and Orchids thrive much less even though the number of canopy plants increases. This is caused by the added competition for resources causing a reduction of the number of Brazil Nut instances by almost 75\% . With a maximum canopy width of eight metres, the Brazil Nut is, by a margin, the plant with the largest canopy. Therefore, even though the aggregate number of canopy plants increases, the actual area of shade decreases, which, as a consequence, causes shade-loving plants to flourish less.\\



Table \ref{tab:results_tropical_big_species_cluster_properties} in appendix \ref{AppendixH} summarizes the vegetation content of each cluster and below are discussed notable characteristics of each.



