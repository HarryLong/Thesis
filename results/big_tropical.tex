\section{Tropical rainforest with fifteen plant species} \label{sec:results_trop_rainforest_big}

Sections \ref{sec:results_trop_rainforest} and \ref{sec:results_alpine} focus on analysing, in detail, the species content of each individual cluster. Because of the level of detail necessary for this analysis and to keep information communicable, the number of species used is limited to five. The system should be scalable, however, and able to cater for a larger number of species. This is the focus of this chapter. To do so, climate data identical to the tropical test run is configured (see table \ref{tab:results_tropical_input_resources}), the same number of clusters produced (ten), but a total of fifteen tropical plant species are configured and used when generating the vegetation distributions (refer to appendix \ref{AppendixE} and \ref{AppendixG} for species details). Because the configuration data and cluster count are identical to that of the tropical test, the clusters remain unchanged and are summarized in figure \ref{fig:results_tropical_cluster_hum_temp_illum} and table \ref{tab:results_tropical_cluster_slope_covarea}.\\

The instance count along with the minimum, maximum and average height of each specie and cluster pair is summarized in appendix \ref{AppendixH}. Similarly to the previous tropical test, these results show that: no plants are able to survive in clusters 3, 5 and 7 primarily because there is too little soil moisture; cluster 8 is ill-suited to all species because of too much soil moisture; clusters 9 and 10 are unsuited because the sun exposure is too little to enable canopy plants to grow but too high to permit shade-loving species to grow without vital shade cover of these canopy plants.\\

When discussing the species content of each cluster below, those present in the tropical test discussed previously (section \ref{sec:results_trop_rainforest}) are ignored as their cluster suitability remains unchanged. A property made apparent by these new tests with added species, however, is that although the suitability of the pre-existing species to each cluster does not change, their instance count and size properties often do. This is a direct consequence of adding more species to the simulation and, therefore, having to redistribute resources. This can work in the species favour, as is the case for Heliconia in cluster one for example. It can also be a disadvantage, as is the case for King of Bromeliads in cluster four. It is difficult to pinpoint why adding species works as an advantage or disadvantage to other species, however, as the cause can be indirect. In clusters five and six, for example, shade-loving species King of Bromeliads and Orchids thrive much less even though the number of canopy plants increases. This is caused by the added competition for resources causing a 75\% reduction in the number of Brazil Nut instances. With a maximum canopy width of eight metres, the Brazil Nut is, by a margin, the plant with the largest canopy. Therefore, even though the aggregate number of canopy plants increases, the actual area of shade decreases, which, as a consequence, causes shade-loving plants to flourish less.\\

Because the soil moisture requirements of the Kapok Tree is too high, it is unable to grow in any of these remaining clusters as their minimum drops below the species configured absolute minimum. The opposite is true for Queens Tear, Poinciana, Fern Begonia and Bahama Wild Coffee as the maximum soil moisture of these clusters is above the maximum allowed for these species and, as such, prevents them from surviving.\\

\paragraph{Cluster One}

Bengal Bamboo, Durian and Bougain Villea are unable to grow in this cluster because the  average sunlight exposure drops below the minimum permitted for these species during winter. Although still negatively impacted by the limited illumination during winder, the more shade-tolerant Coconut Palm is able to develop in this cluster. Limited soil moisture in October and November also slows this species growth. Because of these bottlenecks, the average plant height reaches only thirty percent of the species optimum. Although, with a maximum of just under five metres, some plant instances do reach over eighty percent of this optimum. This points towards inter-plant competition as being a key bottleneck, also.

\paragraph{Cluster Two}

Because of low illumination during winter and a slope of over thirty degrees, no new species are able to grow in this cluster. Note the Heliconia doesn't perform as well in this cluster, even though there are no new species present in the final results. This is because, although they don't survive, a larger set of plant species seed on an annual basis and compete for resources in an attempt to grow. 

\paragraph{Cluster Four}

Bougain Villea is unable to grow in this cluster because, at 414 millimetres, the soil moisture surpasses the species maximum in March.\\
With an average height of over eighty percent of its optimum, Bengal Bamboo thrives extremely well in this cluster. Only slightly too moist soil during the months of February, March and April prevent it from growing to this optimum.\\
With the exception of the soil being insufficiently moist for a couple of months, Coconut Palm and Durian are also well suited to the resources of this cluster. With an average size of only eighteen and thirteen percent of their optimum respectably, this suitability is not reflected in their size. This is caused by intense competition for both soil moisture and sun exposure through the canopies of thriving Bengal Bamboo and Brazil Nut plants.

\paragraph{Cluster Six}

The resources of this cluster is extremely similar to that of cluster six and, as a result, so are the plant distributions. A notable difference, however, is that a decrease in the maximum soil moisture of this cluster permits Bougain Villea to grow. A decrease in the minimum sun exposure of this cluster also causes the growth of Bengal Bamboo to be negatively impacted, resulting in an average size decrease of over ten percent.


