\section{Plant Species}

A database is used to store all plant species and their associated properties. The associated properties are used to determine their ability to grow in given environments and, subsequently, deduce a plausible distribution using the ecosystem simulator. Details about these properties, how they can be edited and new species added is discussed in \textit{Specie Properties} and  \textit{Storing Species} respectively.

\subsection{Specie Properties}

When configuring a new specie it is necessary to specify a set of associated properties. These are used to determine whether or not the specie is suited to a given environment and, if so, by the ecosystem simulator to determine a suitable distribution. Table summarises the properties which are associated with each specie, discussed in detail below.

\begin{table}[]
  \centering
	    \begin{tabular}{|p{4cm}|p{7cm}|p{2cm}|p{0cm}|}
		\hline	
		\textbf{Property} & \textbf{Value} & \textbf{Unit} \\
		\hline
		\textbf{Slope} & Maximum Slope & Degrees \\
		\hline
		\multirow{3}{*}{\textbf{Growth}} & \multicolumn{1}{l|}{Maximum canopy} & \multicolumn{1}{l|}{Centimetres} \\\cline{2-3}
        						   & \multicolumn{1}{l|}{Maximum root size} & \multicolumn{1}{l|}{Centimetres} \\\cline{2-3}
                               & \multicolumn{1}{l|}{Maximum height} & \multicolumn{1}{l|}{Centimetres} \\
		\hline
		\multirow{2}{*}{\textbf{Ageing}} & \multicolumn{1}{l|}{Start of decline} & \multicolumn{1}{l|}{Months} \\\cline{2-3}
        						   & \multicolumn{1}{l|}{Maximum age} & \multicolumn{1}{l|}{Months} \\
		\hline    
		\multirow{2}{*}{\textbf{Seeding}} & \multicolumn{1}{l|}{Maximum seeding distance} & \multicolumn{1}{l|}{Metres} \\\cline{2-3}
        						   & \multicolumn{1}{l|}{Annual seed count} & \multicolumn{1}{l|}{ - } \\
		\hline    
		\multirow{4}{*}{\textbf{Illumination}}
								& \multicolumn{1}{l|}{Start of prime} & \multicolumn{1}{l|}{hours} \\\cline{2-3}
								& \multicolumn{1}{l|}{End of prime} & \multicolumn{1}{l|}{hours} \\\cline{2-3}
								& \multicolumn{1}{l|}{Minimum} & \multicolumn{1}{l|}{hours} \\\cline{2-3}
								& \multicolumn{1}{l|}{Maximum} & \multicolumn{1}{l|}{hours} \\
		\hline   
		\multirow{4}{*}{\textbf{Humidity}}
								& \multicolumn{1}{l|}{Start of prime} & \multicolumn{1}{l|}{millimetres} \\\cline{2-3}
								& \multicolumn{1}{l|}{End of prime} & \multicolumn{1}{l|}{millimetres} \\\cline{2-3}
								& \multicolumn{1}{l|}{Minimum} & \multicolumn{1}{l|}{millimetres} \\\cline{2-3}
								& \multicolumn{1}{l|}{Maximum} & \multicolumn{1}{l|}{millimetres} \\
		\hline    
		\multirow{4}{*}{\textbf{Temperature}}
								& \multicolumn{1}{l|}{Start of prime} & \multicolumn{1}{l|}{degrees} \\\cline{2-3}
								& \multicolumn{1}{l|}{End of prime} & \multicolumn{1}{l|}{degrees} \\\cline{2-3}
								& \multicolumn{1}{l|}{Minimum} & \multicolumn{1}{l|}{degrees} \\\cline{2-3}
								& \multicolumn{1}{l|}{Maximum} & \multicolumn{1}{l|}{degrees} \\
		\hline                                                                           
		\end{tabular}	
\end{table}

\subsubsection{Slope}

Steep slopes cause essential water and soil nutrients to run-off, making them less rich and, therefore, less suited to plant growth. The slope angle also causes larger species to struggle to support their own extensive biomass. For this reason, steeper slopes often cater for smaller plant species (grass, shrub, etc.). To model the effect of slope on given plant species, when configuring a new plant specie an associated \textit{maximum slope} must be specified.

\subsubsection{Growth}

To model the growth of a plant specie in the ecosystem simulator, it is necessary to specify: \textit{Maximum height}, \textit{maximum canopy width} and \textit{maximum root size}. Using this along with the specie's ageing properties (see \ref{subsubsec:ageing_properties}), it is possible to simulate the plants vertical growth (height), horizontal growth (canopy) and root coverage. Determining a plant's height and canopy width is essential in order to determined the shade it project's on other plants during the simulation. Modelling the plant's root growth is used to determine how far the plant can reach in order to fetch vital soil water.\\
A maximum canopy width of zero can be used to model plants with no canopy.

\subsubsection{Ageing} \label{subsubsec:ageing_properties}

Biological life-cycle varies greatly between plant species. Whereas annual and biennials have a fixed lifespan of one and two years respectively, perennial plant species can live far longer. To model the life-cycle of different plant species they must be configured with an associated \textit{start of decline} and \textit{maximum} age. Using these two values, it is possible to simulate a plant getting weaker and becoming more susceptible to domination from surrounding plants.

\subsubsection{Seeding}

It is necessary to replicate the spawning of offspring in the ecosystem simulator for two main reasons: 
\begin{enumerate}
\item \textit{Propagation}: Plants propagate on a terrain by producing new offspring which attempt to spawn and invade different areas. 
\item \textit{Succession}: New plants spawn to later succeed older and weaker plants of the same specie. 
\end{enumerate}

Depending on the specie, plants spawn new offspring by producing seeds or spores. Although biologically different, both can be considered identical for the sole purpose of modelling propagation and succession.\\
The number of seeds/spores that a given plant creates annually is specie-dependent and therefore must accompany the specie configuration.\\
Different species use different techniques to propagate their seeds. For example, some use fruit as a mechanism to propagate using animals digestive systems, some are coated with a sticky mucous and attach to animals fur, some use the wind and some rely solely on gravity to take the seeds to the ground straight below. The seeding mechanism used directly affects the distance which can be achieved between parent and offspring. To emulate this in the system, a \textit{maximum seeding distance} must be configured with each specie.

\subsubsection{Illumination}

Illumination has a big impact on plant growth. Whereas some species strive in the shaded undergrowth, others require direct illumination all year round. To model this, the following illumination values must be configured with each plant specie:

\begin{itemize}
\item \textit{Minimum daily illumination}: The minimum daily illumination, in hours, at which a plant of this specie can survive.
\item \textit{Prime daily illumination range}: The daily illumination range, in hours, which is optimal for the given specie.
\item \textit{Maximum daily illumination}: The maximum daily illumination, in hours, at which a plant of this specie can survive.
\end{itemize}

\subsubsection{Humidity (rainfall)}

Soil water deposited into the soil by either rainfall or existing groundwater is absorbed by plant roots and is vital to its development and survival. Whereas some species have evolved to survive in arid climates with very little water, others require frequent downpours of rain. \\
To simplify water requirement specifications of different plant species, it is necessary to configure the amount of rainfall necessary as if this was the plant's only source of water (i.e no groundwater). The following values must be configured to accurately grasp a species water requirements:

\begin{itemize}
\item \textit{Minimum monthly rainfall}: The minimum monthly rainfall, in millimetres, at which a plant of this specie can survive.
\item \textit{Prime monthly rainfall range}: The monthly rainfall range, in millimetres, which is optimal for the given specie.
\item \textit{Maximum monthly rainfall}: The maximum monthly rainfall, in millimetres, at which a plant of this specie can survive.\end{itemize}

\subsubsection{Temperature}

Another aspect of climates which greatly affect plant growth is temperature. To model a plant species temperature requirements, the following need to be configured:

\begin{itemize}
\item \textit{Minimum temperature}: The minimum temperature, in degrees, at which a plant of this specie can survive.
\item \textit{Prime temperature range}: The temperature range, in degrees, which is optimal for the given specie.
\item \textit{Maximum temperature}: The maximum temperature, in degrees, at which a plant of this specie can survive.
\end{itemize}

\subsection{Storing Species}

All species and associated properties are stored in a database for easy retrieval and filtering. A dedicated tool can be used to interact with the database and add/remove species and edit existing ones (see figure \ref{fig:plant_db_editor}).

\begin{figure}
\center
	\includegraphics[width=\textwidth]{plant_db_editor.png}
	\caption{ Plant database editor tool.}	
	\label{fig:plant_db_editor}
\end{figure}
