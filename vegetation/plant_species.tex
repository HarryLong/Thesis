\section{Plant Species} \label{sec:plant_species}

A database is used to store all plant species and their associated properties, which are used to determine their ability to grow in a given environment and, subsequently, deduce a plausible distribution using the ecosystem simulator. A dedicated tool can be used to interact directly with the database in order to add, remove and edit this data.\\
When configuring a new species it is necessary to specify a set of associated properties. These properties can be split into two main categories: \textit{simulation-based} and \textit{environment-based}. \textit{Simulation-based} properties are those used only by the ecosystem simulator to simulate the growth and spawning of new plants. \textit{Environment-based} properties are those used by the simulator to calculate the strength of the plant but also to determine whether or not it is suited to given environments. Each are discussed below and summarized in table \ref{tab:specie_properties}.\\

\begin{table}[]
  \centering
	    \begin{tabular}{|p{4cm}|p{7cm}|p{2cm}|p{0cm}|}
		\hline	
		\textbf{Property} & \textbf{Value} & \textbf{Units} \\
		\hline
		\multirow{2}{*}{\textbf{Slope}} & \multicolumn{1}{l|}{Start of decline} & \multicolumn{1}{l|}{Degrees} \\\cline{2-3}
        						   & \multicolumn{1}{l|}{Maximum} & \multicolumn{1}{l|}{Degrees} \\
		\hline
		\multirow{3}{*}{\textbf{Growth}} & \multicolumn{1}{l|}{Maximum canopy} & \multicolumn{1}{l|}{Centimetres} \\\cline{2-3}
        						   & \multicolumn{1}{l|}{Maximum root size} & \multicolumn{1}{l|}{Centimetres} \\\cline{2-3}
                               & \multicolumn{1}{l|}{Maximum height} & \multicolumn{1}{l|}{Centimetres} \\
		\hline
		\multirow{2}{*}{\textbf{Ageing}} & \multicolumn{1}{l|}{Start of decline} & \multicolumn{1}{l|}{Months} \\\cline{2-3}
        						   & \multicolumn{1}{l|}{Maximum age} & \multicolumn{1}{l|}{Months} \\
		\hline    
		\multirow{2}{*}{\textbf{Seeding}} & \multicolumn{1}{l|}{Maximum seeding distance} & \multicolumn{1}{l|}{Metres} \\\cline{2-3}
        						   & \multicolumn{1}{l|}{Annual seed count} & \multicolumn{1}{l|}{ - } \\
		\hline    
		\multirow{4}{*}{\textbf{Illumination}}
								& \multicolumn{1}{l|}{Start of prime} & \multicolumn{1}{l|}{hours} \\\cline{2-3}
								& \multicolumn{1}{l|}{End of prime} & \multicolumn{1}{l|}{hours} \\\cline{2-3}
								& \multicolumn{1}{l|}{Minimum} & \multicolumn{1}{l|}{hours} \\\cline{2-3}
								& \multicolumn{1}{l|}{Maximum} & \multicolumn{1}{l|}{hours} \\
		\hline   
		\multirow{4}{*}{\textbf{Humidity}}
								& \multicolumn{1}{l|}{Start of prime} & \multicolumn{1}{l|}{millimetres} \\\cline{2-3}
								& \multicolumn{1}{l|}{End of prime} & \multicolumn{1}{l|}{millimetres} \\\cline{2-3}
								& \multicolumn{1}{l|}{Minimum} & \multicolumn{1}{l|}{millimetres} \\\cline{2-3}
								& \multicolumn{1}{l|}{Maximum} & \multicolumn{1}{l|}{millimetres} \\
		\hline    
		\multirow{4}{*}{\textbf{Temperature}}
								& \multicolumn{1}{l|}{Start of prime} & \multicolumn{1}{l|}{degrees} \\\cline{2-3}
								& \multicolumn{1}{l|}{End of prime} & \multicolumn{1}{l|}{degrees} \\\cline{2-3}
								& \multicolumn{1}{l|}{Minimum} & \multicolumn{1}{l|}{degrees} \\\cline{2-3}
								& \multicolumn{1}{l|}{Maximum} & \multicolumn{1}{l|}{degrees} \\
		\hline                                                                           
		\end{tabular}
	\label{tab:specie_properties}	
	\caption{Summary of the properties which must be configured with each plant species.}
\end{table}

\subsection{Simulation-based Species Properties}

%Growth
To model the growth of a plant species in the ecosystem simulator, it is necessary to specify: \textit{Maximum height}, \textit{maximum canopy width} and \textit{maximum root size}. Using these along with the specie's ageing properties, it is possible to simulate the plants vertical growth (height), horizontal growth (canopy) and root coverage. A plants height and canopy width is also used to determine the shade it projects on other plants during the simulation. Furthermore, the plant's root growth is used to determine how far the plant can reach to fetch soil water. Note that a maximum canopy width of zero can be specified to model plants with no canopy.\\

%Ageing
Biological life-cycle varies greatly between plant species. Whereas annual and biennials have a fixed lifespan of one and two years, respectively, perennial plant species can live far longer. To model the life-cycle of different plant species they must be configured with an associated \textit{age of start of decline} and \textit{maximum age}. Using these two values, it is possible to simulate a plant getting weaker and, therefore, becoming more susceptible to domination from surrounding plants.\\

%Seeding
It is necessary to replicate the spawning of offspring in the ecosystem simulator for two core reasons: \textit{Propagation}: Plants propagate on a terrain by producing new offspring which attempt to spawn and invade different areas. \textit{Succession}: New plants spawn to later succeed older and weaker plants of the same specie.\\ 
The two most common ways for plants to spawn new offspring is through sexual and asexual reproduction. Asexual reproducing species often spawn cloned offspring through budding (e.g potato). Sexual reproducing species, on the other hand, require chromosome exchange between males and females in order to produce unique offspring often propagated via seeds or spores. Although biologically different, both can be considered identical for the sole purpose of modelling propagation and succession. The reproduction characteristics of a given specie which will influence \textit{propagation} and \textit{succession} in the simulation and therefore need to be configured, are the \textit{number of offspring produced annually} and the \textit{maximum distance from source to offspring}.\\

\subsection{Environment-based Species Properties}

%SLOPE
Steep slopes causes essential water and soil nutrients to run-off, making them less rich and, therefore, less suited to plant growth \cite{Kapolka2010}. The slope angle can also cause larger species to struggle in supporting their own biomass. For this reason, steeper slopes often cater better to smaller plant species (grass, shrub, etc.). To model the effect of slope on given plant species, when configuring a new plant species, the \textit{slope of start of decline} and \textit{maximum slope} must be configured.\\

%Illumination
Illumination, soil humidity and temperature also have a great impact on plant growth and survival \cite{Fourcaud2008}.\\
Whereas some species thrive in shaded undergrowth, others require direct illumination all year round. Soil water deposited into the soil by either rainfall or existing groundwater is absorbed by plant roots and is vital to their development and survival. Some species have evolved to survive in arid climates with very little water, others require frequent downpours of rain. To simplify water requirement specifications for different plant species, we ignore groundwater and consider rainfall as the plants only source of water. Some species are able to withstand extremely low temperatures (e.g at high altitudes), others have the ability to survive in extremely hot temperatures (e.g deserts).\\
To configure the illumination, soil humidity and temperature requirements of a given species, it is necessary to configure for each the \textit{minimum}, \textit{prime range} and \textit{maximum}. The minimum represents the minimum illumination (hours), soil humidity (millimetres) or temperature (degrees) necessary for the species survival, the prime range are the values at which the resource is deemed optimal and the maximum is the upper limit after which the plant is unable to survive.

