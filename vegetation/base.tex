\chapter{Vegetation}

Vegetation is an essential part of rural terrains is vegetation. Available resources determine which plant species are able to grow and to what extent they strive in a given environment. Reproducing this link between species and climate is essential to determining suitable vegetation and, subsequently, generating plausible terrains. \\

To determine environments suited for given species, they are configured with associated resource requirements as outlined in section \ref{sec:plant_species}. Given these properties, it is possible to automatically filter out ill-suited plants. Information about this automatic filtering is outlined in section \ref{sec:plant_suitability_filtering}.\\

Although a multitude of plants can grow in a given environment, some will naturally flourish more than others. This can be because resources are more suitable or they have a faster, more aggressive growth rate. To model this intra-species battle for resources and  determine a suitable vegetative state, an ecosystem simulator is used, details of which can be found in section \ref{sec:ecosystem_simulator}.\\

The ecosystem simulator is computationally expensive and can take some time to determine a valid distribution. The simulation time is dependent on the number of plant instances, the simulation area and the timespan. To accelerate the process, the ecosystem simulator is run over a small area and the resulting distribution analysed in order to efficiently reproduce it on larger areas. A caching system is also used to prevent users from having to run the same costly simulation more than once. Information about these features is discussed in section \ref{sec:dist_analysis_and_rep}.
