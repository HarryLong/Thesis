\chapter{Vegetation}

An essential part of rural terrains is vegetation. Available resources determine which plant species are able to grow and to what extent they strive in a given environment. Reproducing this link between species and climate is essential to determine suitable vegetation and subsequently generate plausible terrains. \\

To be able to determine environments suited for given plants, new plants are specified with associated resource requirement properties. Details of which can be found in the \textit{Plant Species} section below.\\
Given these properties, it is possible to automatically filter out ill-suited plants from being suggested to the user. Information about this automatic filtering feature is outlined in the \textit{Resource-based plant filtering} section below.\\

Although a multitude of plants can grow in a given environment, some will naturally strive more than others. This can be because resources are more adequate or they have a faster, more aggressive growth rate. To model this intra-specie battle for resources and  determine a suitable vegetative state, an ecosystem simulator is used. Details of which can be found in the dedicated \textit{Ecosystem Simulator} section below.\\

The ecosystem simulator is computationally expensive and can take some time to determine a valid distribution. The time is dependent on the number of plant instances in the simulation along with the simulation area. To accelerate the process, the ecosystem simulator is run on a small area and the resulting distribution analysed in order to reproduce it on larger area. Information about the distribution analysis and reproduction algorithms can be found in the dedicated \textit{Plant Distribution Analysis and Reproduction} section below.
