Manually creating virtual rural worlds is often a difficult and lengthy task for artists. Plant species that appear, plant distributions and water networks must be deduced such that they realistically reflect the environment being modelled.\\
As virtual worlds grow in size and complexity, climates vary on the terrain itself and a single ecosystem is no longer sufficient to realistically model vegetative content. Consequentially, the task is only getting more difficult for artists.\\
Procedural methods are extensively used in computer graphics to partially or fully automate some tasks and take some of the burden off the user. Input parameters for these procedural algorithms are often unintuitive, however, and their impact on the final results, unclear.  \\
This thesis proposes, implements, and evaluates an approach to procedural generate vegetation and water networks for realistic virtual rural worlds. Rather than placing vegetative content and water networks to reflect the environment being modelled, the work-flow is mirrored and the user models the environment directly by stating the resources it offers. This data is subsequently used as input to procedural algorithms which determine suitable vegetation, plant distributions and water networks. By design, the precision and placeable plant species are easily configurable so any type of environment can be modelled. \\
Overall, the results are promising and the system used to effectively model a multitude of environments. \\