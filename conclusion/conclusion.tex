\chapter{Conclusion} \label{chap:conclusion}

In this work we design, implement and test a new ecosystem generation pipeline, drawing inspiration from the strengths and weaknesses of existing work. The contribution of this work is the concurrent use of clustering, a simulator and a statistical analysis tool to efficiently and realistically place vegetation on any given terrain. Indeed, simulators are computationally expensive and their processing time heavily dependent on level of detail and simulation area. Using radial distribution analysis permits application of an ecosystem simulator solely as an intermediary tool to derive detailed distribution characteristics. Using clustering permits a single ecosystem simulation to realistically populate a large number of areas on the terrain that are deemed similar in terms of resources. Moreso, clustering permits the user to configure, based on their requirements, the balance between efficiency and realism. \\
Extensibility is also a strength of this system; By storing species and their properties in a database, by performing plant suitability filtering and by providing a graphical tool to intuitively add species, the system is, by design, able to cater for any number and type of plant species.\\

\section{Applications} \label{sec:applications}

The main application of this tool would be to quickly and easily create realistic virtual rural worlds for use in video games, simulators or animated movies. Because the inputs are simple and intuitive, it targets not only professional computer generated imagery (CGI) artists but also hobbyist developers. Limited scope did not permit the implementation of a rendering component to generate the final plant renders. The system acts more as a plug-in to existing game engines or rendering software, therefore, to configure vegetation and water content on the terrain.\\

With accurate vegetation and resource configurations, it could also be used to determine suitable vegetation distributions given a modelled environment. This could prove useful for biologists, for example, to estimate plant growth. With the ever-increasing issue of climate change and, as a consequence, investment in climate change predictions, this could be a useful tool to help predict the associated affects on vegetation. Although the tool could potentially be used as is for rough estimation, modelling other resources and existing ones in more detail would dramatically improve accuracy and, therefore, applicability (see section \ref{sec:future_work})\\

\section{Limitations and Future Work} \label{sec:future_work}

The primary limitation of this work is that it cannot be used as an independent virtual world generator but acts more as a tool to be used alongside existing game engines or rendering software. Notably, an essential aspect missing from this work is the ability to generate final renders of the scene with realistic three dimensional plant models. Unfortunately, the time-frame and scope of this work did not permit this.\\

This work focuses heavily on using procedural methods to generate virtual worlds. Unfortunately, a consequence of this is that fine-control over the final content is lost. A useful extension would be to integrate functionality that provides more user control. Examples include: Permitting users to define areas on the terrain on which they want specific species to appear, an eraser tool to remove all vegetation in selected areas and an override tool to manually place plant instances.\\

Procedural methods could be extended upon also to improve the resulting realism. A notable example of this is the ability to automatically place snow. Rainfall and temperature information is already known for each terrain vertex. The altitude at which the temperature drops below zero can therefore easily be determined and used to define the snowline. More so, using the sunlight exposure for each terrain vertex, snow meltage can also be simulated accordingly. This would permit slopes with less sun exposure to retain snow for longer, as they do in reality. \\

Although this work focuses strongly on efficiency in order to maintain continuous interactive feedback, some areas warrant further improvements. As the ecosystem simulator is currently the bottleneck in terms of processing time, much would be gained from improving its efficiency. As simulation area has a strong influence on processing time, one idea would be to make the area vary based on the maximum size attributes of the species it contains. Similarly, the timespan of the simulation could be calculated dynamically based on the ageing properties of contained species. \\
Another way to improve efficiency would be to enhance the radial distribution caching system. As is, the probability that an ecosystem simulator run has already been performed and, therefore, that it can be bypassed entirely, is very slim. Techniques could be used to improve the cache hit, including: Using a range rather than a fixed value for the resources when searching for valid radial distribution data and a background thread which runs simulations during CPU down-time and fills the radial distribution database accordingly.\\

The ecosystem simulator models resources and resource availability in great detail in an attempt to generate plausible results. The accuracy of these simulations could be further improved however and, as a consequence, its application domain broadened. For example, some vital elements that have a big impact on plant development are neglected, including: wind, fire, flooding and soil nutrients.\\
As discussed previously (see section \ref{subsec:humidity_distribution}), soil moisture could be simulated more realistically by modelling the soil with multiple layers. In this way, small plants would not compete for the same soil resources as bigger plants with bigger roots and, therefore, soil reach.\\ 
