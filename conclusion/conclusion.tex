\chapter{Conclusion} \label{chap:conclusion}

This work gives a detailed overview of existing techniques to place vegetation and water networks on virtual terrains. This work also designs, implements and tests a new terrain generation pipeline, drawing inspiration from the strengths and weaknesses of existing work. The novel contribution of this work is the concurrent use of clustering, a simulator and a statistical analysis tool to \textbf{efficiently} and \textbf{realistically} place vegetation on any given terrain. Indeed, simulators are computationally expensive and processing time heavily dependent on level of detail and simulation area. Using radial distribution analysis permits the use of the ecosystem simulator solely as an intermediary tool to grasp detailed distribution characteristics. Using clustering permits a single ecosystem simulation run to realistically populate a large number of areas on the terrain which are deemed similar in terms of resources. More so, clustering permits the user to configure, based on their requirements, the balance between efficiency and realism. \\
Extensibility is also a strength of this system; By storing species and their properties in a database, performing plant suitability filtering and providing a graphical tool to intuitively add species, the system is, by design, able to cater for any number and type of plant species.\\
This chapter discusses the different applications of this work in section ... and, in section ..., the limitations and future work which could extend on this.

\section{Applications} \label{sec:applications}

The main application of this tool would be to quickly and easily create realistic virtual rural worlds to use in video games, simulators or animated movies. Because the inputs are simple and intuitive, it targets not only professional computer generated imagery (CGI) artists but also hobbyist developers. As this system does not generate the final plant renders, it would act as a plug-in to existing game engines or rendering software to configure vegetation and water content on the terrain.\\

With accurate vegetation and resource configurations, it could also be used to determine suitable vegetation and vegetation distribution given a modelled environment. This could prove useful for biologists, for example, to estimate plant growth. With the ever-increasing issue of climate change and, as a consequence, investment in climate change predictions, this could be a useful tool to help predict the associated affects on vegetation. Although the tool could potentially be used as is for rough estimations, modelling more resources and existing ones in more detail would drastically improve accuracy and, therefore, applicability (see section \ref{sec:future_work})\\

\section{Limitations and Future Work} \label{sec:future_work}

The primary limitation of this work is that it cannot be used as an independent virtual world generation pipeline but acts more as a tool to be used alongside existing game engines or rendering software. Notably, an essential part aspect missing from this work is the ability to generate final renders of the scene with realistic three dimensional plant renders. Unfortunately, The time-frame and scope of this work did not permit this.\\

This work focuses heavily on using procedural methods to generate the virtual worlds. Unfortunately, a consequence of this is that fine-control over the final content is lost. A useful extension to this work would be to integrate functionalities that give more user control. Examples include: Permitting users to define areas on the terrain on which they want specific species to appear, an eraser tool to remove all vegetation in selected areas and an override tool to manually place plant instances.\\

Although this work focuses strongly on efficiency in order to maintain continuous interactive feedback, this is not always the case. As the ecosystem simulator is currently the bottleneck of the system in terms of processing time, much would be gained from improving its efficiency. As simulation area has a strong impact on processing time, an idea would be to make the area vary based on the maximum size attributes of the species it contains. \\
Another way to improve efficiency would be to improve the radial distribution caching system. As is, the probability that an ecosystem simulator run has already been performed and, therefore, that it can be bypassed entirely, is very slim. Techniques could be used to improve the cache hit, including: Using a range rather than a fixed value for the resources when searching for valid radial distribution data and a background thread which runs simulations during CPU down-time and fills the radial distribution database accordingly.\\

The ecosystem simulator models resources and resource availability in great detail in an attempt to generate plausible results. The accuracy of these simulations could be further improved however and, as a consequence, its application domain broadened. For example, vital elements which have a big impact on plant development which are neglected include: wind, fire, flooding and soil nutrients.\\
As discussed previously (see section \ref{subsec:humidity_distribution}), soil moisture could be simulated more realistically by modelling the soil as multiple layers. This way, small plants will not compete for the same soil resources as bigger plants with bigger roots and, therefore, soil reach.\\ 
