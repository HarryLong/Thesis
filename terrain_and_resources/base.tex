\chapter{Terrain \& Resources}

The first step in creating virtual worlds is generating the base terrain on which features will be placed. Sequentially, terrain resources with direct influence on content need to be determined. To keep user input to a minimum, this must be done procedurally when possible. \\
This chapter discusses how a system which fitting this requirements was built. The discussion is is split split into the following core sections: \textit{Terrain \& Navigation}, \textit{Resources}, \textit{Rivers \& Streams}, \textit{Water Reserves} and \textit{Results}.\\
\textit{Terrain \& Navigation} discusses how the base terrain is selected and navigated. \\
In order to determine suitable vegetation and river sources, resource information needs to be gathered. How this is done is discussed in the \textit{Resources} section.\\
Essential to the realism of virtual terrains is water placement. This water can take the form of rivers \& streams or water reserves. Techniques used to place such content are discussed in sections \textit{Rivers \& Streams} and \textit{Water bodies} respectively. \\

