\chapter{Terrain \& Resources} \label{chap:terrain_and_resources}

The first step in creating virtual worlds is specifying the base terrain on which features will be placed. Subsequently, terrain resources with direct influence on content are determined. In order to strike a good balance between resulting realism and user experience, procedural methods must be employed when suited.\\
This chapter discusses how a system fitting these requirements was built. The discussion is is split split into the following core sections: \textit{Terrain \& Navigation}, \textit{Resources}, \textit{Rivers \& Streams}, \textit{Water Reserves} and \textit{Results}.\\
\textit{Terrain \& Navigation} discusses how the base terrain is selected and navigated through. \\
In order to determine suitable vegetation and river sources, resource data needs to be specified. How this is done is discussed in the \textit{Resources} section.\\
Essential to the realism of virtual terrains is water placement. This water can take the form of rivers and streams or water reserves. Techniques used to place such content are discussed in the \textit{Rivers and Streams} and \textit{Water bodies} sections, respectively. \\

