\section{Resources}

A core feature of rural landscapes is vegetation and accurately replicating it is therefore critical to the resulting realism of a modelled virtual world. Accurate plant growth modelling is an active area of research as it proves to be an important tool for crop yield optimization \cite{Fourcaud2008}. In their work, Fourcaud et al. \cite{Fourcaud2008} note the importance of modelling the interaction of plants with environmental light, temperature, soil nutrients and water to accurately simulate growth. In order to determine a plausible vegetation layer in our system, \textit{Illumination}, \textit{temperature}, \textit{precipitation}, \textit{soil humidity} and \textit{slope} are modelled. Note that although these resources are deemed essential for accurate plant growth modelling \cite{Fourcaud2008}, it is a simplification as other influential factors such as air quality and air pressure are discarded.\\

\subsection{Illumination}

Sunlight and it's annual variation greatly effects the type and density of vegetation. Whereas some plants prefer habitats with limited sun exposure (e.g lilies), some flourish in fully exposed environments (e.g sunflowers). \\
To determine whether or not a point on the virtual terrain is illuminated at any given time of the year, the system must be able to track the suns trajectory through time.\\

The earth rotates around the sun with an axial tilt, also known as obliquity, of approximately 23.5 degrees (see figure \ref{fig:earth_orbit}). Because of this obliquity, given a position \textit{X} at latitude \textit{L}, the amount of illumination received at \textit{X} in a a 24-hour period will vary during the course of the year (see figure \ref{fig:daylight_variation}). For the northern hemisphere, the day length will be at its maximum during the June equinox and at its minimum during the December solstice. On these days, the earth will be tilted at it's maximum towards and away from the sun respectively.\\

\begin{figure}
\center
	\includegraphics[width=\textwidth]{earth_orbit.jpg}
	\caption{ \textit{Annual orbit of the earth around the sun. Source: \protect\url{http://en.wikipedia.org/wiki/Summer_solstice}}}
	\label{fig:earth_orbit}
\end{figure}

\begin{figure}
\center
	\includegraphics[width=\textwidth]{daylight_variation.png}
	\caption{ \textit{Variation in day length for different latitudes. Source: \protect\url{http://www.physicalgeography.net/fundamentals/6i.html}}}
	\label{fig:daylight_variation}
\end{figure}

In order for the terrain to remain static, when calculating the sun's trajectory the frame of reference is changed to be the earth, around which the sun orbits. To calculate the sun's position at any given time, there are four vital pieces of information that need to be specified by the user: \textit{Latitude}, \textit{Orientation}, \textit{Time of day} and \textit{Month of year}. \\

Specifying the \textit{latitude}, \textit{time of day} and \textit{month of year} is done using sliders which overlay the rendering window. By keeping the render window active during this edit, modifications are clear to the user. When any of these values are changed, the position of the sun is automatically recalculated in real-time. \\

Orientation is displayed to the user at all times with the use of an overlay compass (figure \ref{fig:orientation_control}) inspired by first-person video games. When in orientation edit mode the compass changes to green to provide feedback that the user edit mode is active, at which point the orientation can be modified by using the right/left keyboard keys. Again, all modifications update the sun position in real-time.\\

\begin{figure}
\center
	\includegraphics[width=\textwidth]{orientation_controller.png}
	\caption{ \textit{Orientation controller compass (top of render window). The compass is displayed green to provide feedback to the user that orientation edit mode is active.}}
	\label{fig:orientation_control}
\end{figure}

Given all this information, the first step is to calculate the rotation axis \textit{V$_{RE}$} of the sun at the equinox. This is done using equation \ref{eq:equinox_rotation_axis}. Taking \textit{V$_{RE}$} as the rotation axis for the sun is a simplification. However, the distance between earth's center axis and \textit{V$_{RE}$} is negligible in comparison to the distance between the earth and the sun and is therefore deemed an acceptable simplification.\\

\begin{equation} \label{eq:equinox_rotation_axis}
	V_{RE} = R(V_{N}, \textit{-L}, V_{E})
\end{equation}
where: \textit{V$_{RE}$} is the rotation axis of the sun at the equinox; \textit{V$_{N}$} is the north-facing vector passing through the terrain center; \textit{V$_{E}$} is the east-facing vector passing through the terrain center; \textit{L} is the latitude of the terrain; \textit{R(V$_{a}$,\textit{a},V$_{b}$}) is the resulting vector after rotating V$_{a}$ by \textit{a} degrees around V$_{b}$\\

\textit{V$_{RE}$} is the rotation axis for the sun at the March and December equinox. During the equinox, axis tilt has no effect on daytime duration as the tilt is not directed away or towards the sun. At this point, latitude alone is the determinant of daytime duration. In order to calculate the rotation axis V$_{R}$(m) of the sun at month \textit{m}, axis tilt must be taken into consideration by further rotating $V_{RE}$ using equation \ref{eq:all_month_rotation_axis}.

\begin{equation} \label{eq:all_month_rotation_axis}
	V_{R}(m) = R(V_{RE}, a_{m}, V_{E})
\end{equation}
where: \textit{V$_{R}(m)$} is the rotation axis of the sun at month \textit{m};\textit{V$_{RE}$} is the rotation axis of the sun at the equinoxes (equation \ref{eq:equinox_rotation_axis});\textit{V$_{E}$} is the east-facing vector passing through the terrain center;\textit{a$_{m}$} is the rotation angle calculated using equation \ref{eq:rotation_angle_calculation};\textit{R(V$_{a}$,\textit{a},V$_{b}$}) is the resulting vector after rotating V$_{a}$ by \textit{a} degrees around V$_{b}$\\

\begin{equation} \label{eq:rotation_angle_calculation}
	a_{m} = -tilt_{max} + |6-m| \times tilt_{monthly} $$\\
$$
tilt_{monthly} = tilt_{max}/3
\end{equation}

where: \textit{$a_{m}$} is the rotation angle at month \textit{m}; \textit{$tilt_{max}$} is the maximum axis tilt of the earth (~23.5 degrees)

The time of day, \textit{t} is then used to determine the amount the sun is rotated around the rotation axis \textit{V$_{R}$(m)}. With a full rotation being performed every 24 hours.

\subsubsection{Calculating Illumination}

A point on the terrain is illuminated if there is a direct path from it to the sun with no intersections with other points on the terrain. To test for this on the terrain ray casting is performed from each vertex position towards the sun to check whether or not it intersects with other points on the terrain.\\
This process can be lengthy, however, as a ray casting operations needs to be performed for each individual terrain vertex. In order to accelerate this process, a spherical hierarchical acceleration structure is used. This hierarchical acceleration structure employs a tree structure to iteratively search for smaller intersection areas. Using this acceleration structure, illumination can be calculated for 4 million vertices in just over 2 seconds (figure \ref{fig:illumination_calculation_time}). As shown in figure \ref{fig:illumination_calculation_time}, there is a linear relationship between vertex count and calculation time. \\

\begin{figure}
\center
	\includegraphics[width=\textwidth]{illumination_calculation_time.png}
	\caption{ \textit{Illumination calculation time based on vertex count. Analysis performed for terrains with dimensions: 256 by 256, 512 by 512, 1024 by 1024, 1536 by 1536 and 2048 by 2048.} }
	\label{fig:illumination_calculation_time}
\end{figure}

An important element which will influence vegetation on the terrain is the variation in the hours of illumination received daily during the course of the year. To determine the illumination received on a given day, the illumination calculation is used by iterating through each hour of the day consecutively and determining whether or not a vertex \textit{V} is illuminated. In order to reduce the number of illumination calculations to perform when determining the variation of the illumination throughout the year, the illumination is calculated for the fifteenth day of every month rather than for every day of the year.

To illustrate the daily illumination on the terrain to the user for a given month \textit{m}, an \textit{illumination overlay} can be enabled which darkens and lights up terrain vertices proportionally to the amount of light received (see figure \ref{fig:overlay_daily_illumination}).

\begin{figure}
\center
	\includegraphics[width=\textwidth]{daily_illumination_overlay.png}
	\caption{ \textit{Daily illumination overlay. The brighter the area, the more illumination it receives throughout the day.} }
	\label{fig:overlay_daily_illumination}
\end{figure}

\subsection{Temperature}

Whereas tropical climates often have relatively constant temperatures throughout the year, others, such as the continental climate, are characterized by a strong variation between minimum and maximum annual temperatures. Only plants which are able to survive at both extremes can grow, which is why temperature and its variation has a significant impact on vegetation. For modelling purposes, it is acceptable to assume that the minimum temperature, \textit{T$_{min}$}, occurs in the middle of winter and the maximum temperature, \textit{T$_{max}$}, occurs in the middle of summer. Interpolation can be performed to determine the temperature at any time between these two dates.\\

Properties which are necessary to model temperature in the system include: \textit{Altitude}, \textit{Temp$_{december}$}, \textit{Temp$_{june}$} and \textit{Lapse rate}. The \textit{altitude} of each point on the terrain is calculated automatically based on the properties of height-map loaded. Temp$_{december}$ and Temp$_{june}$ represent both extremes of the temperature spectrum at zero meters of altitude and need to be configured by the user. The \textit{lapse rate} defines the decrease in temperature with altitude. Although this changes depending on atmospheric conditions, the default is configured to a value of 6.4 degrees Celsius for each km increase in altitude. This is accepted as the average atmospheric lapse rate under normal atmospheric conditions \protect\footnotemark.\\
\footnotetext{\url{http://en.wikipedia.org/wiki/Lapse_rate}}

Given this information, the temperature is calculated for any point on the terrain given the month and altitude using equation \ref{eq:temp_calculation}. 

\begin{equation} \label{eq:temp_calculation}
	T(a,m) = T_{december} + ( \frac{6 - |6-m|}{6} \times (T_{june} - T_{december}))
\end{equation}

where:\textit{T(a,m)} is the temperature at altitude \textit{a} and month \textit{m}; \textit{$T_{december}$} is the temperature at zero meters in December; \textit{$T_{june}$} is the temperature at zero meters in June.\\

Calculating the temperature for 4 million vertices (2048 by 2048 terrain ) takes approximately 2 seconds. Figure \ref{fig:temperature_calculation_time} points towards a linear relationship between vertex count and temperature calculation time.

\begin{figure}
\center
	\includegraphics[width=\textwidth]{temperature_calculation_time.png}
	\caption{ \textit{Temperature calculation time based on vertex count. Analysis performed for terrains with dimensions: 256 by 256, 512 by 512, 1024 by 1024, 1536 by 1536 and 2048 by 2048.} }
	\label{fig:temperature_calculation_time}
\end{figure}

An overlay can be enabled to provide a graphical overview of the temperature at different locations on the terrain for the selected month (see figure \ref{fig:temp_overlay}).

\begin{figure}
\center
	\includegraphics[width=\textwidth]{temp_overlay.png}
	\caption{ \textit{Temperature overlay. Colour spectrum ranging from blue (cold) to red (hot). As can be seen, the temperature drops with altitude.}}
	\label{fig:temp_overlay}
\end{figure}

\subsection{Precipitation} \label{sec:precipitation}

Precipitation is a core part of climate classification and, consequentially, plant life. Arid climates have very limited annual precipitation and this is a bottleneck for organic life. Tropical climates, on the other hand, where precipitation is plentiful, have an abundance of vegetation. There are two important properties of precipitation that are modelled: \textit{Quantity} and \textit{Intensity}. The \textit{quantity}, often measured in mm, defines the amount rain that falls. The intensity, often measured in mm/h defines the rate at which it falls.\\

The user must configure \textit{quantity} and \textit{intensity} values for each month of the year. A custom input dialogue was implemented in an attempt to make this as user-friendly as possible (\ref{fig:rainfall_input}).

\begin{figure}
\center
	\includegraphics[width=\textwidth]{rainfall_input.png}
	\caption{ \textit{Specifying monthly precipitation and precipitation intensity. The user can enter values manually (using the input fields) or interact directly with the graph.}}
	\label{fig:rainfall_input}
\end{figure}

\subsection{Soil Humidity}

When rain falls onto the terrain, a certain portion of it is absorbed into the soil to provide the plant's roots with the necessary nutrients. The portion which is absorbed depends on the type of soil. Rocky soils, for example, have limited water retention and will result in larger water build-up and potentially run-off. In this work, the soil humidity is a measure, in mm, of the rainfall which is absorbed by the soil for each given month. This is determined using the precipitation information outlined above (\ref{sec:precipitation}) along with the \textit{Soil Infiltration Rate}.\\

Soil humidity, also referred to as soil moisture, is most commonly measured as the volumetric water content in the soil, as a percentage \cite{Corps1980}. Calculating the volumetric ratio of water to soil would require soil depth data for each terrain vertex which, due to scope, is not represented in our system. Millimetres of rainfall was deemed an adequate measure, however, as the water requirements of different plant species are often stated in millimetres of rainfall, therefore providing a good correlation between available and required resource. 

\subsubsection{Soil Infiltration Rate}\label{subsubsec:soil_infiltration_rate}

The \textit{soil infiltration rate} is a measure of the quantity of water which can be absorbed in a given period. If the rainfall intensity exceeds the soil infiltration rate, it will result in water stagnation (on a flat surface) or water run-off. This rate is correlated with the type of soil and approximate values for some soil types can be found in table \ref{tab:soil_types}.

\begin{table}[h]
  \centering
	    \begin{tabular}{|p{5cm}|p{8cm}|}
  	    \hline	
  	    \textbf{Soil-type} & \textbf{Infiltration rate (mm/hour)} \\
		\hline
		\textbf{sand} & $<$ 30 \\
		\hline
		\textbf{sandy loam} & 20-30 \\
		\hline
		\textbf{loam} & 10-20 \\
		\hline
		\textbf{clay loam} & 5-10 \\
		\hline
		\textbf{clay} & 1-5 \\
		\hline
		\end{tabular}
		\caption{\textit{Soil infiltration rates for different soil types} \protect\footnotemark}
	  \label{tab:soil_types}
\end{table}
\footnotetext{\url{http://www.fao.org/docrep/s8684e/s8684e0a.htm}}


Specifying the soil infiltration rate can be done using three distinct tools: \textit{Filling}, \textit{Slope-based} and through a \textit{Painting interface}.\\
Using the \textit{filling} tool, the user can fill the entire terrain with a configured infiltration rate.
The \textit{slope-based} tool permits the user to configure a slope above which the soil infiltration rate will be zero. This is an efficient tool for modelling cliff faces or simply areas where water run-off is too severe. The \textit{painting interface} enables user to paint a soil infiltration on the terrain directly using a common brush tool found in basic paint applications. The size of the brush can be configured using the scroll-wheel and the painted soil-infiltration using a dedicated slider controller (\ref{fig:soil_infiltration_controls}).

These tools can be used interchangeably and users are encouraged to do so. Configuring the soil infiltration of the terrain in figure \ref{fig:soil_infiltration_controls}, for example, was performed in approximately three minutes using all three tools as follows:
\begin{enumerate}
\item The filling tool was used to fill the entire terrain with an infiltration rate of 25.\\
\item The slope-based tool was used to set to zero the infiltration rate of all areas with a slope above 30 degrees.\\
\item The painting-tool was used to paint an infiltration rate of zero on the flat area to the right to cater for a water build-up (reservoir, lake, ocean, etc.).\\
\end{enumerate}

\begin{figure}
\center
	\includegraphics[width=\textwidth]{soil_infiltration_rate_controls.png}
	\caption{ \textit{Editing the soil infiltration rate on the terrain. Top left are the controllers for the \textit{filling} and \textit{slope-based} tools. On the right is the slider to configure the soil infiltration rate of the \textit{paint brush}.} }
	\label{fig:soil_infiltration_controls}
\end{figure}

Given the \textit{soil infiltration rate}, \textit{monthly precipitation} and \textit{monthly average precipitation intensity} for each vertex on the terrain, it is possible to calculate the quantity of rainfall that is actually absorbed by the soil for each month. This represents the soil humidity in our system and is calculating using equation \ref{eq:monthly_soil_humidity}. 

\begin{equation} \label{eq:monthly_soil_humidity}
	S_{h}(R_{q},R_{i},S_{ir}) = 
	\frac{S_{ir}}{R_{i}} \times R_{q}
\end{equation}
where:\textit{$S_{h}$} is the soil humidity for a given month, in mm; \textit{$R_{q}$} is the monthly rainfall quantity, in mm; \textit{$R_{i}$} is the average monthly rainfall intensity, in millimetres per hour;\textit{$S_{ir}$} is the soil infiltration rate, in millimetres per hour.\\

Intuitively, equation \ref{eq:monthly_soil_humidity} calculates the proportion of the total rainfall which is able to be absorbed by the soil given the rainfall intensity and absorption rate of the soil.\\

To accelerate the process, the soil humidity is calculated in parallel for each vertex on the GPU. By doing so, 4 million vertices (2048 by 2048 terrain) can be processed in 104 milliseconds (figure \ref{fig:soil_humidity_calculation_time}). There is a linear relationship between vertex count and calculation time (figure \ref{fig:soil_humidity_calculation_time}).

\begin{figure}
\center
	\includegraphics[width=\textwidth]{soil_humidity_calculation_time.png}
	\caption{ \textit{Soil humidity calculation time based on vertex count. Analysis performed for terrains with dimensions: 256 by 256, 512 by 512, 1024 by 1024, 1536 by 1536 and 2048 by 2048.} }
	\label{fig:soil_humidity_calculation_time}
\end{figure}


\subsubsection{Weighted Monthly Soil Humidity Calculation}

Monthly soil humidity does not take into account the humidity of previous months and therefore fails to model water retention in the soil. By retaining water in the soil, the drought caused by an arid month can be counteracted to some extent by precipitation in previous months. To model this, a moving weighted average soil humidity is calculated for each month using equation \ref{eq:weighted_avg_monthly_humidity_calculation}.

\begin{equation} \label{eq:weighted_avg_monthly_humidity_calculation}
	S_{wh}(m) =  (\frac{1}{2} \times S_{h}(m)) + (\frac{1}{3} \times S_{h}(m-1)) + (\frac{1}{6} \times S_{h}(m-2))
\end{equation}
where: \textit{$S_{wh}(m)$} is the weighted soil humidity at month \textit{m}, in mm; \textit{$S_{h}(m)$} is the soil humidity at month \textit{m}, in mm. \\

Similarly to the monthly humidity, the GPU is used to accelerate the calculation process. By doing so, 4 million vertices (2048 by 2048 terrain) can be processed in 178 milliseconds (figure \ref{fig:weighted_soil_humidity_calculation_time}). There is a linear relationship between vertex count and calculation time (figure \ref{fig:weighted_soil_humidity_calculation_time}).

\begin{figure}
\center
	\includegraphics[width=\textwidth]{weighted_soil_humidity_calculation_time.png}
	\caption{ \textit{Weighted soil humidity calculation time based on vertex count. Analysis performed for terrains with dimensions: 256 by 256, 512 by 512, 1024 by 1024, 1536 by 1536 and 2048 by 2048.} }
	\label{fig:weighted_soil_humidity_calculation_time}
\end{figure}

A visual overlay can be displayed to give an overview of both the monthly soil humidity and the weighted average monthly soil humidity on the terrain (figure \ref{fig:soil_humidity_overlay}).

\begin{figure}
\center
	\includegraphics[width=\textwidth]{soil_humidity_overlay.png}
	\caption{ \textit{Soil humidity terrain overlay. Colour spectrum ranging from black (zero humidity) to blue (standing water).} }
	\label{fig:soil_humidity_overlay}
\end{figure}

\subsection{Slope}

Slopes cause soil to be lost due to the effects of gravity. This loss in soil and therefore soil nutrients cause a bottleneck for plant growth \cite{Kapolka2001}
. This is clearly visible in nature, where only smaller plants (shrub, grass, etc.) grow on steeper landscapes. To determine suitable vegetation given the terrain relief, it is therefore important to model slope.\\

This is calculated in real-time using the GPU (104 milliseconds for 4 million vertices) and the relation between vertex count and calculation time is linear (\ref{fig:slope_calculation_time})

\begin{figure}
\center
	\includegraphics[width=\textwidth]{slope_calculation_time.png}
	\caption{ \textit{Slope calculation time based on vertex count. Analysis performed for terrains with dimensions: 256 by 256, 512 by 512, 1024 by 1024, 1536 by 1536 and 2048 by 2048.}  }
	\label{fig:slope_calculation_time}
\end{figure}

To better visualize the slope throughout the terrain, a \textit{slope overlay} can be enabled (see figure \ref{fig:slope overlay}).

\begin{figure}
\center
	\includegraphics[width=\textwidth]{slope_overlay.png}
	\caption{ \textit{Slope visual overlay. Colour spectrum ranging from black (zero degree slope) to white (90 degree slope).} }
	\label{fig:slope overlay}
\end{figure}