\section{Resources}

Two core features of rural landscapes are vegetation and water networks. Accurately replicating these features is therefore critical to the resulting realism of the modelled virtual world.\\
The combination and annual variance of temperature, illumination and precipitation have a direct effect on vegetation and water networks. Therefore, to procedurally determine the vegetation and water networks, available resources must first be resolved. This section will focus on the approach taken to determine available resources throughout the terrain. Due to scope, not all resources could be modelled and this work focuses on: \textit{Illumination}, \textit{temperature}, \textit{precipitation}, \textit{soil humidity},  and \textit{slope}. 

\subsection{Illumination}

Illumination and annual variation thereof greatly effects the type and density of vegetation that can grow. Whereas some plants prefer habitats with limited sunlight exposure (lilies, etc.), some strive in fully exposed environments (sunflowers, etc.). \\
To determine whether or not a point on the virtual terrain is illuminated at any given time of the year, the system must be able to track the suns trajectory through time. How this is done and how the terrain illumination is calculated accordingly is discussed below.

\subsubsection{Calculating the suns trajectory}

The earth rotates around the sun with an axial tilt, also known as obliquity, of approximately 23.5 degrees (see figure \ref{fig:earth_orbit}). Because of this obliquity, given a position \textit{X} at latitude \textit{L}, the amount of illumination received at \textit{X} in a a 24-hour period will vary during the course of a year (see figure \ref{fig:daylight_variation}). For the northern hemisphere, the day length will be at it's maximum during the June equinox and at it's minimum during the December equinox. At these moments in time, the earth will be tilted at it's maximum towards and away from the sun respectively.\\

\begin{figure}
\center
	\includegraphics[width=\textwidth]{earth_orbit.jpg}
	\caption{ Earth orbiting the sun \protect\footnotemark}
	\label{fig:earth_orbit}
\end{figure}
\footnotetext{\url{http://en.wikipedia.org/wiki/Summer_solstice}}

\begin{figure}
\center
	\includegraphics[width=\textwidth]{daylight_variation.png}
	\caption{ Variation in day length for different latitudes \protect\footnotemark}
	\label{fig:daylight_variation}
\end{figure}
\footnotetext{\url{http://www.physicalgeography.net/fundamentals/6i.html}}

In order for the terrain to remain static, when calculating the sun's trajectory the frame of reference is changed to be the earth, around which orbits the sun. To calculate the sun's position at any given time, there are four vital pieces of information that need to be specified by the user: \textit{Latitude}, \textit{Orientation}, \textit{Time of day} and \textit{Month of year}. \\

Specifying the \textit{latitude}, \textit{time of day} and \textit{month of year} is done using sliders which overlay the rendering window (figure \ref{fig:lat_and_time_control}). By keeping the render window visible during this edit, modifications are clear to the user. When any of these values are changed, the position of the sun is automatically recalculated in real-time. \\

\begin{figure}
\center
	\includegraphics[width=\textwidth]{controlling_time_and_latitude.png}
	\caption{ Time (top) and latitude (bottom) controllers }
	\label{fig:lat_and_time_control}
\end{figure}

Orientation is displayed to the user at all times with the use of an overlay compass (figure \ref{fig:orientation_control}) inspired by first-person video games. When in "orientation edit mode" the compass changes to green to inform the user edit mode is active, at which point the orientation can be modified by using the right/left keyboard keys. Again, all modifications update the sun position in real-time.\\

\begin{figure}
\center
	\includegraphics[width=\textwidth]{orientation_controller.png}
	\caption{ Orientation controllers}
	\label{fig:orientation_control}
\end{figure}

Given all this information, the first step is to calculate the rotation axis \textit{V$_{RE}$} of the sun at the equinox. This is done using equation \ref{eq:equinox_rotation_axis}. Taking \textit{V$_{RE}$} as the rotation axis for the sun is a simplification. However, the distance between earth's center axis and \textit{V$_{RE}$} is negligible in comparison to the distance between the earth and the sun and is therefore deemed an acceptable simplification.\\

\begin{equation} \label{eq:equinox_rotation_axis}
	V_{RE} = R(V_{N}, \textit{-L}, V_{E})
\end{equation}
where:
\begin{itemize}
\item \textit{V$_{RE}$} is the rotation axis of the sun at the equinox.\\
\item \textit{V$_{N}$} is the north-facing vector passing through the terrain center.\\
\item \textit{V$_{E}$} is the east-facing vector passing through the terrain center.\\
\item \textit{L} is the latitude of the terrain.\\
\item \textit{R(V$_{a}$,\textit{a},V$_{b}$}) is the resulting vector after rotating V$_{a}$ by \textit{a} degrees around V$_{b}$\\
\end{itemize}

\textit{V$_{RE}$} is the rotation axis for the sun at the March and December equinox. During the equinox, axis tilt has no effect on daytime duration as the tilt is not directed away or towards the sun. At this point, only latitude is the determinant factor of daytime duration. In order to calculate the rotation axis V$_{R}$(m) of the sun at month \textit{m}, axis tilt must be taken into consideration by further rotating $V_{RE}$ using equation \ref{eq:all_month_rotation_axis}.

\begin{equation} \label{eq:all_month_rotation_axis}
	V_{R}(m) = R(V_{RE}, a_{m}, V_{E})
\end{equation}
where:
\begin{itemize}
\item \textit{V$_{R}(m)$} is the rotation axis of the sun at month \textit{m}.\\
\item \textit{V$_{RE}$} is the rotation axis of the sun at the equinoxes (\ref{eq:equinox_rotation_axis}.\\
\item \textit{V$_{E}$} is the east-facing vector passing through the terrain center.\\
\item \textit{a$_{m}$} is the rotation angle calculated using equation \ref{eq:rotation_angle_calculation} .\\
\item \textit{R(V$_{a}$,\textit{a},V$_{b}$}) is the resulting vector after rotating V$_{a}$ by \textit{a} degrees around V$_{b}$\\
\end{itemize}

The time of day, \textit{t} is then used to determine the amount the sun is rotated around the rotation axis \textit{V$_{R}$(m)}. With a full rotation being performed every 24 hours.

\begin{equation} \label{eq:rotation_angle_calculation}
	a_{m} = -tilt_{max} + |6-m| \times tilt_{monthly} $$\\
$$
tilt_{monthly} = tilt_{max}/3
\end{equation}

where:
\begin{itemize}
\item \textit{$a_{m}$} is the rotation angle at month \textit{m}
\item \textit{$tilt_{max}$} is the maximum axis tilt of the earth (~23.5 degrees)
\end{itemize}

\subsubsection{Calculating Illumination}

A point on the terrain is illuminated if there is a direct path from it to the sun with no intersections with other points on the terrain. To test for this on the terrain ray casting is performed from each vertex position towards the sun to check whether or not it intersects with other points on the terrain.\\
A spherical hierarchical acceleration structure is used to accelerate the ray casting process. This hierarchical acceleration structure acts as a tree structure to iteratively search for smaller intersection areas. Calculating the illumination for each vertex on a 1024 by 1024 terrain takes less than 5 seconds.

\subsubsection{Illumination overlay} \label{subsub:illumination}

To visually present terrain illumination to the user an \textit{illumination overlay} can be selected which renders an overlay on-top of the terrain (see figure \ref{fig:overlay_illumination})

\begin{figure}
\center
	\includegraphics[width=\textwidth]{illumination_overlay.png}
	\caption{ Illumination overlay. Illuminated areas are coloured white. Shaded areas are coloured black. }
	\label{fig:overlay_illumination}
\end{figure}

\subsubsection{Daily Illumination}

An important factor for plant growth is the hours of illumination received throughout the day. The illumination calculation is used (\ref{subsub:illumination}) to determine this by going through each hour of the day consecutively and determining whether or not a vertex \textit{V} is illuminated. 

\subsubsection{Daily Illumination overlay}

Similarly to the illumination, a visual presentation of the daily illumination can be enabled as observed in figure \ref{fig:overlay_daily_illumination}.

\begin{figure}
\center
	\includegraphics[width=\textwidth]{daily_illumination_overlay.png}
	\caption{ Daily illumination overlay. The brighter the area, the more illumination it receives throughout the day. }
	\label{fig:overlay_daily_illumination}
\end{figure}

\subsection{Temperature}

Whereas tropical climates often have relatively constant temperatures throughout the year, others, such as the continental climate, are characterized by a strong variation between minimum and maximum annual temperatures. Only plants which are able to survive at both extremes can grow which is why temperature and variation thereof has a big impact on vegetation. For modelling purposes, it is acceptable to assume that the minimum temperature, \textit{T$_{min}$}, occurs in the middle of winter and the maximum temperature, \textit{T$_{max}$}, occurs in the middle of summer. Interpolation can be performed to determine the temperature at any time between these two dates.\\

Properties which affect temperature and are necessary to model temperature in the system are: \textit{Altitude}, \textit{Temp$_{december}$}, \textit{Temp$_{june}$} and \textit{Lapse rate}. The \textit{altitude} of each point on the terrain is calculated automatically based on the loaded height-map. Temp$_{december}$ and Temp$_{june}$ represent both extremes of the temperature spectrum at zero meters of altitude and need to be configured by the user. The \textit{lapse rate} defines the decrease in temperature with altitude. Although this changes depending on atmospheric conditions, the default is configured to be 6.4 °C for each km of altitude gained. This is accepted as the average atmospheric lapse rate under normal atmospheric conditions \protect\footnotemark.
\footnotetext{\url{http://en.wikipedia.org/wiki/Lapse_rate}}. 

Given this information, the temperature is calculated for any point on the terrain at given the month \textit{m} and its altitude \textit{A}, using equation \ref{eq:temp_calculation}.

\begin{equation} \label{eq:temp_calculation}
	T(a,m) = T_{december} + ( \frac{6 - |6-m|}{6} \times (T_{june} - T_{december}))
\end{equation}

where:
\begin{itemize}
\item \textit{T(a,m)} is the temperature at altitude \textit{a} and month \textit{m}.
\item \textit{$T_{december}$} is the temperature at zero meters in December.
\item \textit{$T_{june}$} is the temperature at zero meters in June.
\end{itemize}

\subsubsection{Temperature overlay}

Similarly to the illumination, an overlay can be enabled to illustrate the temperature at different points on the terrain (\ref...
