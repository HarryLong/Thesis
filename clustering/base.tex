\chapter{Clustering}

Each point on the terrain has a number of associated resource properties (summarised in table \ref{tab:point_resources}) which are used to depict suitable vegetation and, using an ecosystem simulator, a distribution of said vegetation. This simulator is computationally expensive, however, and running it for each terrain vertex is infeasible. As such, to make the number of ecosystem simulations which need to be run manageable, clustering is performed on the terrain based on the resources associated to the individual points. The type of clustering algorithm used is \textit{K-Means Clustering}, discussed in the dedicated \textit{K-Means Clustering} section. To accelerate the clustering algorithm and strive for real-time results, it is implemented to run on the GPU, details of which can be found in the \textit{GPU Implementation} section. To conclude, the performance and results of the clustering algorithm is discussed.

\begin{table}[h]
  \centering
	    \begin{tabular}{|p{6cm}|p{3cm}|p{6cm}|}
		\hline	
  	    \textbf{Resource} & \textbf{Count} & \textbf{Comments} \\
  	    \hline	
  	    Slope & 1 & - \\
		\hline
  	    Temperature & 12 & Temperature for each month \\
		\hline
  	    Illumination & 12 & Illumination for each month \\
		\hline
  	    Soil Humidity & 12 & Soil humidity for each month \\
		\hline
		\end{tabular}
		\caption{Resource properties associated with a single point on the terrain.}
	  \label{tab:point_resources}
\end{table}



